\documentclass[11pt,center,lualatex]{beamer}

\usepackage{xltxtra}	% XeLaTeX
\usepackage{float}
\usepackage{graphicx}
\usepackage{wrapfig}	% Wrap text around figs
\usepackage{lscape}
\usepackage{rotating}	% Sideways figure
\usepackage{hyperref}	% Hyperlinks
\usepackage{caption}	% Hyperlinks to float top
\usepackage{subcaption}
\usepackage{units}		% dB unit
\usepackage[a4paper,width=150mm,top=25mm,bottom=25mm]{geometry}	% margins
\usepackage{tikz}
\usepackage[section]{placeins}	% Don't let floats float before sections
\usepackage{listings}
\usepackage{amsmath}  % align equations
\usepackage{biblatex}

\usepackage{fancyhdr} % headers-footers
\usepackage{mathtools}
% Tables
\usepackage{array}
\usepackage{booktabs}
% for having numbers aligned to the decimal point
\usepackage{siunitx}


%-------------------------------------Custom commands----------------------------------------------------------
  \usepackage{amsfonts}
  \def \IN{\mathbb N} \def \IZ{\mathbb Z} \def \IQ{\mathbb Q} \def \IR{\mathbb R} \def \IC{\mathbb C}
  \def \NN{\mathbb N} \def \ZZ{\mathbb Z} \def \QQ{\mathbb Q} \def \RR{\mathbb R} \def \CC{\mathbb C}
%-------------------------------GREEK LETTERS DEFINITION-------------------------------------------------------
  \newcommand{\gra}{{\alpha}} \newcommand{\grb}{{\beta}} \newcommand{\grg}{{\gamma}} \newcommand{\grd}{{\delta}}
  \newcommand{\gre}{{\epsilon}} \newcommand{\grz}{{\zeta}} \newcommand{\grh}{{\eta}} \newcommand{\gru}{{\theta}}
  \newcommand{\gri}{{\iota}} \newcommand{\grk}{{\kappa}} \newcommand{\grl}{{\lambda}} \newcommand{\grm}{{\mu}}
  \newcommand{\grn}{{\nu}} \newcommand{\grj}{{\xi}} \newcommand{\gro}{{\rm o}} \newcommand{\grp}{{\pi}}
  \newcommand{\grr}{{\rho}} \newcommand{\grs}{{\sigma}} \newcommand{\grt}{{\tau}} \newcommand{\gry}{{\upsilon}}
  \newcommand{\grf}{{\phi}} \newcommand{\grx}{{\chi}} \newcommand{\grc}{{\psi}} \newcommand{\grv}{{\omega}}
  \newcommand{\grA}{{\rm A}} \newcommand{\grB}{{\rm B}} \newcommand{\grG}{{\Gamma}} \newcommand{\grD}{{\Delta}}
  \newcommand{\grE}{{\rm E}} \newcommand{\grZ}{{\rm Z}} \newcommand{\grH}{{\rm H}} \newcommand{\grU}{{\Theta}}
  \newcommand{\grI}{{\rm I}} \newcommand{\grK}{{\rm K}} \newcommand{\grL}{{\Lambda}} \newcommand{\grM}{{\rm M}}
  \newcommand{\grN}{{\rm N}} \newcommand{\grJ}{{\Xi}} \newcommand{\grO}{{\rm O}} \newcommand{\grP}{{\Pi}}
  \newcommand{\grR}{{\rm R}} \newcommand{\grS}{{\Sigma}} \newcommand{\grT}{{\rm T}} \newcommand{\grY}{{\rm Y}}
  \newcommand{\grF}{{\Phi}} \newcommand{\grX}{{\rm X}} \newcommand{\grC}{{\Psi}} \newcommand{\grV}{{\Omega}}
%---------------------------------------------------------------------------------------------------------------

\setromanfont[Mapping=tex-text]{Linux Libertine O}
\setsansfont[Mapping=tex-text]{DejaVu Sans}
\setmonofont[Mapping=tex-text]{DejaVu Sans Mono}

%\geometry{margin=2.5cm}
\pagestyle{fancy}

\graphicspath{ {figures/} }
\addbibresource{references.bib}

% Alignment and placing of subtables
%   https://tex.stackexchange.com/questions/294589/alignment-and-placing-of-subtables
\usepackage{booktabs,subcaption,amsfonts,dcolumn}
\newcolumntype{d}[1]{D..{#1}}
\newcommand\mc[1]{\multicolumn{1}{c}{#1}} % handy shortcut macro

\renewcommand{\figurename}{Σχήμα}
\renewcommand{\tablename}{Πίνακας}
\renewcommand{\contentsname}{Περιεχόμενα}

\usetikzlibrary{arrows.meta,automata,decorations.pathmorphing,decorations.markings,
  backgrounds,positioning,fit,shapes.misc,
  graphs,arrows
}
\tikzset{node/.style={
    rectangle,
    minimum size= 1.2cm,
    thick, draw=black,
    top color=white!50!yellow!10!,
    bottom color=yellow!50!white!50!,
    font=\ttfamily
}}

\usepackage{fancybox}
\usepackage{minibox}
%\usepackage[backend=biber, style=numeric, citestyle=authoryear]{biblatex}
\usepackage[absolute,overlay]{textpos}
\usetheme{metropolis}

\newcommand{\mli}[1]{\mathit{#1}}

%---		BIBLIOGRAPHY		---%
\bibliography{../references.bib}
%\setbeamertemplate{bibliography item}[text]
\renewcommand*{\bibfont}{\footnotesize}

\definecolor{light-gray}{gray}{0.86}
\title{\huge{Υψηλού επιπέδου υλοποίηση των αλγορίθμων Hierarchical Temporal Memory σε Julia}}
\author{Κωνσταντίνος Σαμαράς-Τσακίρης}
\date{13 Ιουνίου 2019}

\begin{document}

\begin{frame}%{\maketitle}
  \titlepage
\end{frame}

% Intro
\begin{frame}{Το πρόβλημα}
  \begin{block}{Επάρκεια}
	Προσαρμόζεται ένα μοντέλο στα δεδομένα. Αρκεί για να τα περιγράψει ικανοποιητικά; Εξάγει όλη την πληροφορία;
  \end{block}

  \pause
  this $φ f \rightarrow d$ Υπόλοιπα \alert{iid}

  \pause
  Έστω, για γραμμικά μοντέλα, \alert{λευκός θόρυβος}
\end{frame}

\begin{frame}{Portmanteau}
  Λευκός θόρυβος $\Rightarrow$ μηδενική αυτοσυσχέτιση $\rho$ ($\tau > 0$)

  \pause
  Με δεδομένη $r$, έλεγχος:
  $$ H_0: \rho_\tau = 0 \qquad H_1: \rho_\tau \neq 0$$

  \pause
  \begin{block}{Ljung \& Box portmanteau statistic}
	$$ Q = n(n+2) \sum^{K}_{\tau=1} \frac{r_\tau^2}{n-\tau} $$

	$ \sim \mathcal{X}^2 $ με k βαθμούς ελευθερίας

	$\rightarrow$ Απόρριψη $H_0$ αν $ Q > \mathcal{X}^2_{k;1-a} $
  \end{block}

  \pause
  \vspace{1em} \centering
  Αδυναμίες;
\end{frame}

\begin{frame}{AR vs. MA underfit}
  \begin{block}{Portmanteau μερικής αυτοσυσχέτισης \tiny{\cite{monti1994}}}
    Αντικατάσταση στο $Q$ της ολικής με μερική αυτοσυσχέτιση:
    $$ Q_\phi = n(n+2) \sum^{K}_{\tau=1} \frac{\phi_{\tau\tau}^2}{n-\tau} $$
  \end{block}

  \pause
  \begin{itemize}
    \item Μικρή τάξη AR: \emph{εξίσου ισχυρό} με $Q$
    \item Μικρή τάξη \alert{MA}: \emph{πιο ευαίσθητο} από $Q$
  \end{itemize}
\end{frame}

\begin{frame}{Βελτιώσεις $Q$}
  Ιδέα:
  \begin{itemize}
    \item Mικρότερο βάρος στις αργότερες αυτοσυσχετίσεις
    \item Υπολογίζονται με λιγότερα δείγματα
  \end{itemize}

  \visible<2>{
  \begin{block}{Σταθμισμένα στατιστικά Portmanteau \tiny{\only<2>{\cite{fisher2011}}}}
    $$ Q_w = n(n+2) \sum_{k=1}^m \frac{m-k+1}{m} \frac{\hat{r}_k^2}{n-k} $$
    $$ M_w = n(n+2) \sum_{k=1}^m \frac{m-k+1}{m} \frac{\hat{\phi}_{\tau\tau}^2}{n-k} $$

    Κατανομή: γραμμικός συνδυασμός $\mathcal{X}^2$
  \end{block}}
\end{frame}

\begin{frame}{Βελτιώσεις $Q$}
  \footcite{monti1994}
\end{frame}


\begin{frame}{Εποχικές χρονοσειρές}
  \only<1-2>{
  Περιοδικά υπόλοιπα
  \pause
  \begin{itemize}
    \item Ένδειξη ανεπαρκούς μοντέλου
    \item Αυτοσυσχέτιση όχι ιδιαίτερα ευαίσθητη στην περιοδικότητα
  \end{itemize}}

  \pause
  \only<3>{
  \begin{block}{Periodogram! \tiny{\cite{sekar2010}}}
    Παρόμοιο με μετασχηματισμό Fourier:
    $$ I(f_i) = \frac{2}{n} \left[\left(\sum^n_{t=1} \alpha_t \cos 2\pi f_i t \right)^2 + \left(\sum^n_{t=1} \alpha_t \sin 2\pi f_i t \right)^2 \right] $$
    όπου $\alpha_t$ τα δείγματα της χρονοσειράς, $f_i = \frac{i}{n}$ συχνότητα
  \end{block}}

  \pause \pause
  \begin{block}{Cumulative periodogram}
    Το κανονικοποιημένο ολοκλήρωμα του periodogram:
    $$ C(f_j) = \frac {\sum^j_{i=1} I(f_i)} {n \hat{\sigma}_a^2} $$
  \end{block}

  \pause
  \begin{description}[Λευκός θόρυβος]
	\item[Λευκός θόρυβος] $C(f_j)$ ευθεία γραμμή μεταξύ (0,0) και (0.5,1)
	\item[Περιοδικότητα] Απόκλιση από την ευθεία
	\begin{itemize}
	  \item Έλεγχος σημαντικότητας: Kolmogorov-Smirnov 1-sample test
	\end{itemize}
  \end{description}
\end{frame}

\nocite{*}
\begin{frame}[plain,allowframebreaks]
  \printbibliography
\end{frame}

\end{document}
