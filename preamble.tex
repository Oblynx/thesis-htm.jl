\usepackage{xltxtra}	% XeLaTeX
\usepackage{fontspec}
\usepackage{float}
\usepackage{graphicx}
\usepackage{svg}
\usepackage{wrapfig}	% Wrap text around figs
%\usepackage{lscape}
\usepackage{rotating}	% Sideways figure
\usepackage{amsmath}  % align equations
\usepackage{centernot}
\usepackage{hyperref}	% Hyperlinks
\usepackage[font=footnotesize]{caption}	% Hyperlinks to float top
\usepackage{subcaption}
\usepackage{cleveref}
%\usepackage{units}		% dB unit
\usepackage[a4paper,left=25.4mm,right=25.4mm,top=25mm,bottom=25mm]{geometry}	% margins
\usepackage[section]{placeins}	% Don't let floats float before sections

%\usepackage{jlcode}
\usepackage{listings}
\usepackage{fancyhdr} % headers-footers
\usepackage{mathtools}
% Tables
\usepackage{array}
\usepackage{booktabs}

\usepackage{microtype}
\usepackage{polyglossia}
\setdefaultlanguage[variant=mono]{greek}
\setotherlanguage{english}
\enablehyphenation
\usepackage{hyphsubst}
\usepackage{alphabeta}  % greek letters in maths mode
\usepackage{csquotes}
\usepackage[
backend=biber,
style=ieee,
sorting=ynt
]{biblatex}
\usepackage{xcolor}
\usepackage{enumitem}

%\renewcommand{\mkbegdispquote}[2]{\itshape}

%-------------------------------------Custom commands----------------------------------------------------------
  \usepackage{amsfonts}
  \def \IN{\mathbb N} \def \IZ{\mathbb Z} \def \IQ{\mathbb Q} \def \IR{\mathbb R} \def \IC{\mathbb C}
  \def \NN{\mathbb N} \def \ZZ{\mathbb Z} \def \QQ{\mathbb Q} \def \RR{\mathbb R} \def \CC{\mathbb C}
  \newcommand{\norm}[1]{\left\lVert#1\right\rVert}
  \newcommand{\notimplies}{%
    \mathrel{{\ooalign{\hidewidth$\not\phantom{=}$\hidewidth\cr$\implies$}}}}
  \newcommand{\Simplies}[1][]{\DOTSB\;\Longrightarrow^{#1}\;}
  \newcommand{\simplies}[1][]{\DOTSB\;\Longrightarrow_{#1}\;}

%\setromanfont[Mapping=tex-text]{Linux Libertine O}
%\setsansfont[Mapping=tex-text]{DejaVu Sans}
%\setmonofont[Mapping=tex-text]{DejaVu Sans Mono}
\setromanfont{Noto Serif}
\setsansfont{Noto Sans}
\setmonofont{Noto Mono}

%\geometry{margin=2.5cm}
\pagestyle{fancy}

% Alignment and placing of subtables
%   https://tex.stackexchange.com/questions/294589/alignment-and-placing-of-subtables
\usepackage{booktabs,subcaption,dcolumn}
\newcolumntype{d}[1]{D..{#1}}
\newcommand\mc[1]{\multicolumn{1}{c}{#1}} % handy shortcut macro

\renewcommand{\figurename}{Σχήμα}
\renewcommand{\tablename}{Πίνακας}
\renewcommand{\contentsname}{Περιεχόμενα}
\renewcommand{\appendixname}{Παραρτήματα}


\usepackage{tikz}
\newcommand*\circled[1]{\tikz[baseline=(char.base)]{
    \node[shape=circle,draw,inner sep=2pt] (char) {#1};}}
\usetikzlibrary{arrows.meta,automata,decorations.pathmorphing,decorations.markings,
  backgrounds,positioning,fit,shapes.misc,
  graphs,arrows
}
\tikzset{node/.style={
    rectangle,
    minimum size= 1.2cm,
    thick, draw=black,
    top color=white!50!yellow!10!,
    bottom color=yellow!50!white!50!,
    font=\ttfamily
}}

\usepackage{letltxmacro}	% Safely renew commands (normal \let \renew might cause infinite loops if used on robust commands -- see https://tex.stackexchange.com/questions/47351/can-i-redefine-a-command-to-contain-itself)
\LetLtxMacro{\oldemph}{\emph}
\renewcommand{\emph}[1]{\textcolor{blue}{\oldemph{#1}}}

\usepackage{unicode-math}
\setmathfont{Latin Modern Math}

% for having numbers aligned to the decimal point
% also engineering notation
\usepackage{siunitx}
\sisetup{output-exponent-marker=\ensuremath{\mathrm{e}}}

\usepackage[titletoc]{appendix}

%\captionsetup[subfigure]{subrefformat=simple,labelformat=simple}
%\renewcommand\thesubfigure{(\alph{subfigure})}
