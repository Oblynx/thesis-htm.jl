\section{Στόχος της εργασίας}

  Σε αυτήν την εργασία μελετάται μια αλγοριθμική θεωρία νοημοσύνης, η \textit{Hierarchical Temporal Memory}.
  Βασικοί αλγόριθμοι της θεωρίας διατυπώνονται σε υψηλού επιπέδου γλώσσα με εκφραστική περιεκτικότητα,
  αποσκοπώντας στην καταπολέμηση της βασικής μορφής πολυπλοκότητας που απαντά σε αυτό το μοντέλο:
  την \textit{εκφραστική πολυπλοκότητα} \parencite{chazelleNaturalAlgorithmsInfluence}.

  Με προγραμματιστική διατύπωση πιστή στη μαθηματική διατύπωση των αλγορίθμων, το έργο αυτό φιλοδοξεί να θεμελιώσει και να
  διευκολύνει την περαιτέρω μελέτη ενός συστήματος που, πέρα από το επιστημονικό του ενδιαφέρον,
  προσφέρεται για την αντιμετώπιση δύσκολων προβλημάτων τεχνητής νοημοσύνης.
  Η Hierarchical Temporal Memory βρίσκεται σε φάση ενεργούς έρευνας και ανάπτυξης.
  Καθώς οι αλγόριθμοι που την περιγράφουν εξελίσσονται, μια πλατφόρμα που επιτρέπει ταχύ πειραματισμό σε επεκτάσεις και εναλλακτικές ιδέες
  μπορεί να διαδραματίσει σημαντικό ρόλο στην περαιτέρω μελέτη της θεωρίας.

\section{Θεμελιώνοντας την έννοια της νοημοσύνης}

  Κάθε φυσικό σύστημα καθορίζεται από τους περιορισμούς στα σύνορά του.
  Ομοίως ο άνθρωπος καθορίζεται από την αλληλεπίδραση με το περιβάλλον του.
  Στην προσπάθεια να κατανοήσουμε τον άνθρωπο, το πώς λειτουργεί, το γιατί δρα με συγκεκριμένο τρόπο,
  σταθμό αποτελεί η κατανόηση της συμπεριφοράς που καλούμε \textit{νοημοσύνη}.

  Η συμπεριφορά είναι παρατηρήσιμη και μας δίνει μια οπτική στην εσωτερική κατάσταση, στις κρυφές μεταβλητές ενός συστήματος,
  οπότε ίσως αποτελεί το σημείο όπου μπορούμε να πιάσουμε το νήμα της αναζήτησης.
  Είναι όμως ικανοποιητικό να χαρακτηρίσουμε τη νοημοσύνη συμπεριφορά;

  Ο όρος νοημοσύνη χαίρει ευρείας ερμηνείας, ευρισκόμενος στο σταυροδρόμι πολλών επιστημονικών πεδίων, από την ψυχολογία μέχρι επιστήμη υπολογιστών
  \parencite{leggCollectionDefinitionsIntelligence2007}.
  Όλες μάλιστα οι πιο συγκεκριμένες ερμηνείες προσπίπτουν στο ιδιαίτερα ασαφές νόημα του όρου στην καθομιλουμένη.
  Ίσως μπορούμε να ξεμπλέξουμε για πρακτικούς σκοπούς αυτό το κουβάρι, παρατηρώντας ότι νοημοσύνη σίγουρα επιδεικνύουν ζωντανοί οργανισμοί ως εξελικτικό χαρακτηριστικό,
  ως εργαλείο στη διαρκή προσαρμογή τους στο επίσης δυναμικό τους περιβάλλον.

\subsection*{Νοημοσύνη ως προσαρμογή}

  Η νοημοσύνη λοιπόν, ως προσαρμογή, προκύπτει από τη σχέση του οργανισμού με το περιβάλλον του.
  Αν ο οργανισμός προσαρμόζεται ταιριάζοντας υλικά του κατασκευάσματα στο περιβάλλον του, η νοημοσύνη επεκτείνει αυτήν τη δημιουργικότητα, επιτρέποντάς του να πειραματιστεί εικονικά.
  \parencite[σελ 3]{piagetOriginsIntelligenceChildren1952}.
  Ας ακολουθήσουμε όμως τη σκέψη του ψυχολόγου Jean Piaget στο ζήτημα.

  Προσαρμογή είναι μια δυναμική διαδικασία αλληλεπίδρασης του οργανισμού με το περιβάλλον που περιλαμβάνει 2 στάδια: \textit{αφομοίωση} και \textit{συμβιβασμό}.
  Έστω ότι ο οργανισμός μπορεί να περιγραφεί με μια σειρά εσωτερικών μεταβλητών $\{a,b,c\}$ και εξωτερικών στοιχείων του περιβάλλοντος $\{x,y,z\}$,
  που συνδέονται μεταξύ τους με κάποιες διαδικασίες, ορίζοντας ένα μοντέλο (schema):
  \begin{align*}
    a + x &\rightarrow b\\
    b + y &\rightarrow c\\
    c + z &\rightarrow a
  \end{align*}

  Η \textit{αφομοίωση} συνίσταται στην ικανότητα του οργανισμού να ενσωματώνει τα στοιχεία του περιβάλλοντος στις εσωτερικές του καταστάσεις και να συνεχίζει αυτές τις διαδικασίες.
  Μια αλλαγή όμως στο περιβάλλον, έστω $x\rightarrow x'$, αποτελεί πρόκληση.
  Είτε ο οργανισμός δεν προσαρμόζεται, που σημαίνει ότι ο κύκλος σπάει και ο οργανισμός παύει να επιτελεί κάποια λειτουργία του,
  είτε προσαρμόζεται, τροποποιώντας υποχρεωτικά το μοντέλο του για να \textit{συμβιβαστεί} με τη νέα εξωγενή πραγματικότητα ($b\rightarrow b'$):
  \begin{align*}
    a  + x' &\rightarrow b'\\
    b' + y  &\rightarrow c\\
    c  + z  &\rightarrow a
  \end{align*}

  Σύμφωνα με αυτήν την περιγραφή, \textit{προσαρμογή} είναι η ισορροπία της \textit{αφομοίωσης} ($a+x\rightarrow b$...) με το \textit{συμβιβασμό} ($x\rightarrow x' \Rightarrow b\rightarrow b'$).

  Η προηγούμενη περιγραφή ισχύει εξίσου για τη \textit{νοημοσύνη}. Νοημοσύνη είναι αφομοίωση, στο βαθμό που συμπεριλαμβάνει όλα τα εμπειρικά δεδομένα στη δομή της.
  Είναι όμως και συμβιβασμός, καθώς κατά τη διαρκή αφομοίωση αποκρίνεται στην πρόκληση των περιβαλλοντικών αλλαγών με τροποποίηση του μοντέλου που περιγράφει τον κόσμο.

  Η διανοητική προσαρμογή λοιπόν, όπως κάθε προσαρμογή, συνίσταται από ένα μηχανισμό αφομοίωσης και συμπληρωματικού συμβιβασμού που διατηρούνται σε διαρκή ισορροπία.
  Ένα μυαλό προσαρμοσμένο στην πραγματικότητα είναι αυτό που δε δέχεται πια προκλήσεις στο νοητικό του μοντέλο για τον κόσμο,
  που δε χρειάζεται να τροποποιήσει περαιτέρω το μοντέλο αυτό για να εξηγήσει την εξελισσόμενη πραγματικότητα \parencite[σελ 5-7]{piagetOriginsIntelligenceChildren1952}.

\subsection*{Ένας πρακτικός ορισμός}

  Από το συμπεριφορικό ορισμό του Turing στο γνωστό "Turing test" μέχρι το μηχανιστικό ορισμό του Piaget,
  και με πολλές στάσεις ενδιάμεσα στο \cite[Lenat][]{lenatThresholdsKnowledge1991} και στο \cite[Minsky][]{minskySocietyMind1988},
  η πρακτικότητα της έννοιας τίθεται υπό αμφισβήτηση.

  Ένας άξονας του ορισμού είναι η σχέση της νοημοσύνης με τη λογική. Στο \cite{wangCognitiveLogicMathematical}
  ο Wang χωρίζει τα συλλογιστικά συστήματα σε 3 κατηγορίες:
  \begin{itemize}
    \item Αμιγώς αξιωματικά. Όλες οι λογικές προτάσεις προκύπτουν από τα αξιώματα, χαρακτηριστικό παράδειγμα η ευκλείδια γεωμετρία.
    \item Μερικώς αξιωματικά. Η γνώση δεν επαρκεί σε όλες τις περιπτώσεις και υπάρχει μηχανισμός προσαρμογής, όπως στα ασαφή συστήματα.
    \item Μη αξιωματικά. Χτίζονται με βάση την υπόθεση ότι η γνώση ή οι πόροι δεν επαρκούν για οποιοδήποτε συλλογισμό.
  \end{itemize}

  Προτείνει λοιπόν ότι η απαίτηση νοημοσύνης ισοδυναμεί με την απαίτηση μη αξιωματικού συλλογιστικού συστήματος,
  λόγω της ανεπάρκειας πληροφορίας και πόρων.

  Προσθέτοντας άλλο ένα έρεισμα στη συζήτηση, η νοημοσύνη συσχετίζεται άμεσα με τη δημιουργικότητα \parencite{benedekIntelligenceCreativityCognitive2014}.
  Δημιουργικότητα είναι η ικανότητα παραγωγής καινοτόμων και χρήσιμων ιδεών,
  επομένως συνδέεται με την τέλεση των προαναφερθέντων συμβιβασμών.
  \bigskip

  Σε αναζήτηση ενός χρήσιμου ορισμού στα πλαίσια αυτής της εργασίας, μπορούμε να στραφούμε στον ορισμό του \cite[Wang][]{wangWorkingDefinitionIntelligence1995}:

  \begin{displayquote}
    Νοημοσύνη είναι η ικανότητα ενός συστήματος επεξεργασίας πληροφορίας να προσαρμόζεται
    στο περιβάλλον του με ανεπαρκή γνώση και πόρους.
  \end{displayquote}

  Αναφερόμαστε σε σύστημα επεξεργασίας πληροφορίας, για να μπορούμε να μελετήσουμε την εσωτερική του κατάσταση και αλληλεπίδραση
  με το περιβάλλον αφηρημένα, σε αντιδιαστολή με ένα πρόβλημα π.χ. ρομποτικής.
  Το σύστημα έχει μια γλώσσα εισόδου και εξόδου, με την οποία εκφράζονται τα ερεθίσματα του περιβάλλοντος και οι δράσεις του συστήματος.
  Το σύστημα συνήθως έχει κάποιο σκοπό για τον οποίο παράγει δράσεις σύμφωνα με τη γνώση του,
  και για την επεξεργασία του δαπανά περιορισμένους πόρους.

  Η προσαρμογή μπορεί να ερμηνευθεί όπως προηγουμένως, ή πιο συνοπτικά ως ότι το σύστημα μαθαίνει από τις εμπειρίες του.

  Ο περιορισμός της ανεπαρκούς γνώσης και πόρων σημαίνει ότι το σύστημα υπόκειται σε αυτές τις συνθήκες:
  \begin{itemize}
    \item \textit{Περιορισμένο ως προς τους υπολογιστικούς του πόρους}
    \item Λειτουργεί σε \textit{πραγματικό χρόνο}
    \item Καλύπτει όλο το πεδίο των εκφράσιμων στη γλώσσα του εισόδων και εξόδων (\textit{δεν υπάρχουν άκυρες είσοδοι/έξοδοι})
  \end{itemize}

  Σύμφωνα με αυτόν τον ορισμό νοημοσύνη είναι μια ισχυρή μορφή προσαρμογής.

  Οι παραπάνω περιορισμοί μας οδηγούν προς την εξέταση \textit{συστημάτων ροής (streaming)}, με \textit{ανθεκτικότητα σε σφάλματα}
  και με \textit{απόδοση που να κλιμακώνεται με τους διαθέσιμους πόρους} --- υπό την προϋπόθεση της \textit{διαρκούς προσαρμογής σε νέες συνθήκες} του περιβάλλοντος.

  Η αυλαία σηκώνεται για να παρουσιαστεί το υπό μελέτη σύστημα: \textbf{Hierarchical Temporal Memory (HTM) της Numenta}.

\section{Γιατί μελετούμε την HTM;}

  Ο πρακτικός ορισμός της νοημοσύνης επιβάλλει περιορισμούς στο τι σύστημα θα θεωρήσουμε ότι επιδεικνύει χαρακτηριστικά νοημοσύνης.
  Για παράδειγμα, τα συστήματα εμπειρογνωμόνων δεν πληρούν αρκετές από τις προδιαγραφές, ενώ πολλά συστήματα νευρωνικών
  δικτύων επιβλεπόμενης μάθησης που βρίσκονται τώρα σε χρήση επίσης προσπίπτουν τουλάχιστον στην προδιαγραφή της διαρκούς προσαρμοστικότητας.

\subsection{Φυσικοί αλγόριθμοι}

  Ακολουθώντας τη λογική του Chazelle \cite{chazelleNaturalAlgorithmsInfluence}, η επιτυχία της φυσικής του 20ου αιώνα είναι σε μεγάλο βαθμό η επιτυχία της μαθηματικής έκφρασης.
  Με ένα μικρό σύνολο εξισώσεων μπορούμε να περιγράψουμε τι συμβαίνει στο φυσικό κόσμο.
  Αυτή η διαπίστωση αν μη τι άλλο σκιαγραφεί τις αρχές που διέπουν το φυσικό κόσμο:
  συμμετρία και κανονικότητα, η αμβροσία της συνήθους μαθηματικής διατύπωσης.
  Αν αυτή η παρατήρηση ήταν καθολική, τα ίδια εργαλεία θα επαρκούσαν για να περιγράψουν όλα τα επιστημονικά πεδία.

  Η βιολογία διέπεται προφανώς από τους ίδιους φυσικούς νόμους και συνίσταται στην εφαρμογή τους επανειλημμένα στο βάθος των αιώνων.
  Παρόλο που οι αρχές είναι οι ίδιες, φαινομενολογικά η υπόθεση εργασίας είναι αντίστροφη:
  καθε φαινόμενο είναι ειδικό και ξεχωριστό, αλλοιώσιμο υπό κάθε μετασχηματισμό, εκτός από ορισμένες περιπτώσεις που μπορούν να κατηγοριοποιηθούν μαζί.
  Γιατί οι αρχές επιτρέπουν καταστάσεις διακλάδωσης στα στοιχειώδη συστήματα που περιγράφουν και διάσπαση της συμμετρίας \parencite{bradingSymmetrySymmetryBreaking2017}.
  Έτσι, ο αναγωγισμός δε συνεπάγεται εποικοδομητισμό \parencite{andersonMoreDifferent1972}.
  Με μια κομψή έκφραση: \textit{η ιστορία είναι ο μεγάλος διασπαστής της συμμετρίας} \parencite{chazelleNaturalAlgorithmsInfluence}.

  Δίχως την απλοποιητική επιρροή της συμμετρίας, αυτή η οπτική γωνία έχει να αντιμετωπίσει μια μορφή πολυπλοκότητας διαφορετική
  από αυτήν που συνήθως ορίζουμε στην επιστήμη υπολογιστών: \textit{εκφραστική πολυπλοκότητα}.
  Είναι ωφέλιμο αντιστοίχως να χρησιμοποιήσουμε και διαφορετική γλώσσα για τη μελέτη
  αυτών των φαινομένων: \textit{τους (φυσικούς) αλγορίθμους} \parencite{chazelleNaturalAlgorithmsInfluence}.
  Παραδείγματα φυσικών αλγορίθμων εξερευνούνται στα \cite{patonComputationCellsTissues2013,adamatzkyAdvancesPhysarumMachines2016}
  \medskip

  Συνδέοντας τον πρακτικό ορισμό της νοημοσύνης και τη χρησιμότητα των αλγορίθμων για να περιγράψουν \textit{εκφραστικά πολύπλοκα} φαινόμενα,
  μπορούμε να επιχειρήσουμε τη μελέτη της νοημοσύνης με μια αλγοριθμική της θεώρηση.
  Αυτό είναι το βασικό κίνητρο για το μοντέλο που μελετά αυτή η εργασία, τη \textit{Hierarchical Temporal Memory} (HTM).

\subsection{HTM ως μοντέλο του εγκεφαλικού νεοφλοιού}

\subsubsection{Εποπτική εγκεφαλική ανατομία}

  Ο ανθρώπινος εγκέφαλος χωρίζεται σε διακριτές δομές με διαφορές τόσο στο μορφολογικό, όσο και στο λειτουργικό επίπεδο.
  Ένας χρήσιμος τέτοιος διαχωρισμός συνίσταται από τα εξής τμήματα:
  \begin{itemize}
    \item Πρόσθιος/διάμεσος εγκέφαλος, που περιλαμβάνει τα εγκεφαλικά ημισφαίρια, τους θαλάμους, τους ιπποκάμπους κ.α.
    \item Παρεγκεφαλίδα, που εντοπίζεται στο πίσω μέρος του κρανίου κάτω από τα ημισφαίρια
    \item Εγκεφαλικό στέλεχος, που εντοπίζεται επίσης προς τα πίσω και είναι η προέκταση του νωτιαίου μυελού
  \end{itemize}

  Σε αυτό το σημείο αξίζει μία πολύ συνοπτική επισκόπηση των εγκεφαλικών δομών, για την καλύτερη κατανόηση της HTM.

  Το εγκεφαλικό στέλεχος είναι το εξελικτικά αρχαιότερο τμήμα του εγκεφάλου και σχετίζεται με βασικές ομοιοστατικές λειτουργίες.
  Το μεταιχμιακό σύστημα στη βάση των ημισφαιρίων αναπτύχθηκε πριν από περίπου 250 εκατομμύρια χρόνια στα θηλαστικά
  και μία από τις βασικές του λειτουργίες είναι η ρύθμιση των συναισθημάτων.
  Αυτές οι δομές έχουν μορφολογία πυρηνική, δηλαδή τα σώματα των νευρώνων τους συγκεντρώνονται σε σφαιροειδείς δομές
  από τις οποίες εκτείνονται οι άξονές τους.

  Η επιφάνεια των εγκεφαλικών ημισφαιρίων είναι ο εγκεφαλικός φλοιός, με το μεγαλύτερο μέρος του να αποτελεί το νεοφλοιό,
  όπου οι νευρώνες (τα σώματα) είναι δομημένοι σε 6 επίπεδα, και σε μικρό μέρος τον αλλοφλοιό, που έχει 3 επίπεδα νευρώνων.
  Οι άξονες των νευρώνων φεύγουν από το επίπεδο του φλοιού σαν καλώδια σε πλακέτα.
  Ο φλοιός είναι το εξελικτικά πιο σύγχρονο τμήμα του εγκεφάλου, και ειδικά ο νεοφλοιός, που απαντά μόνο σε θηλαστικά.
  Εδώ εντοπίζονται λειτουργίες σχετικές με "ανώτερη συλλογιστική" και αφηρημένη σκέψη.
  Η παρεγκεφαλίδα επίσης έχει τη μορφή φλοιού, αλλά είναι διακριτή από τα ημισφαίρια. Σχετίζεται τουλάχιστον με τη
  ρύθμιση λεπτών κινήσεων. Και οι δύο αυτές δομές που οργανώνονται σε φλοιούς, αντί για πυρήνες, μοιράζονται ένα
  γεωμετρικό πλεονέκτημα: το μέγεθός τους μπορεί να κλιμακωθεί ευκολότερα.
  Αποτέλεσμα της κλιμάκωσης του μεγέθους τους είναι οι αναδιπλώσεις στην επιφάνεια του ανθρώπινου εγκεφάλου.
  Στον άνθρωπο ο νεοφλοιός αποτελεί περίπου τα 3/4 όλου του εγκεφάλου.

  Στον τομέα της μηχανικής μάθησης εμφανίζονται 3 βασικά μοντέλα μάθησης: επιβλεπόμενη, μη επιβλεπόμενη και ενισχυτική.
  Υπάρχουν επιχειρήματα στη βιβλιογραφία \parencite{doyaWhatAreComputations1999} ότι 3 από τις εγκεφαλικές δομές που περιγράφηκαν εφαρμόζουν
  αυτά τα 3 μοντέλα αντίστοιχα: η παρεγκεφαλίδα, ο εγκεφαλικός φλοιός και τα βασικά γάγγλια.

  Σημαντικά ανατομικά στοιχεία του φλοιού είναι η οργάνωση των νευρώνων του σε επίπεδα και οι αναδρομικές συνδέσεις μεταξύ τους.
  Έχει επίσης παρατηρηθεί ότι η πλαστικότητα των νευρικών συνάψεων στο φλοιό ακολουθεί κανόνα Hebbian:
  ισχυροποιούνται όταν το προσυναπτικό ερέθισμα συσχετίζεται με μετασυναπτική δραστηριότητα και εξασθενούν αλλιώς,
  χτίζοντας την αιτιώδη σχέση μεταξύ προσυναπτικής και μετασυναπτικής δραστηριότητας.
  Διατυπώνεται έτσι η υπόθεση ότι ο φλοιός μαθαίνει με μη επιβλεπόμενο τρόπο να οργανώνει την εξωγενή και εσωτερική πραγματικότητα σε έννοιες,
  να δημιουργεί συμβολικές αναπαραστάσεις για μεταβλητές κατάστασης.

  Με δεδομένη αυτήν τη βασική ανατομία, μπορούμε να διαπιστώσουμε τι σκοπεύει να μοντελοποιήσει η HTM.

\subsubsection{Στόχος της HTM}

  Η θεωρία της Hierarchical Temporal Memory αναπτύσσεται από την ερευνητική εταιρία Numenta, που δηλώνει το διττό της στόχο ως εξής:
  καταρχήν, την αλγοριθμική μοντελοποίηση της λειτουργίας του ανθρώπινου εγκεφαλικού νεοφλοιού,
  και ως συνέπεια τη μελέτη των εφαρμογών της θεωρίας τούτης ως σύστημα τεχνητής νοημοσύνης.
  Επομένως το μοντέλο που μελετά αυτή η εργασία δεν έχει σχεδιαστεί κατά κύριο λόγο ως σύστημα τεχνητής νοημοσύνης,
  αλλά ως \textit{θεωρία της λειτουργίας του ανθρώπινου εγκεφαλικού νεοφλοιού, περιορισμένη από βιολογικά δεδομένα}.
  Παρόλα αυτά, εδώ δε θα συζητηθεί η νευροεπιστημονική πιστότητα του μοντέλου, μονάχα οι βιολογικές αρχές στις οποίες βασίζεται.

  Σύμφωνα με τα παραπάνω ανατομικά στοιχεία, μοντελοποιώντας το νεοφλοιό η HTM δεν αποτελεί πλήρες μοντέλο
  του ανθρώπινου εγκεφάλου. Δεν προσφέρεται για παράδειγμα για μελέτη των συναισθημάτων ή των βασικών ομοιοστατικών
  μηχανισμών, αλλά μόνο των "ανώτερων συλλογιστικών". Η επιλογή αυτή δεν είναι τυχαία. Χάρη στην εξελικτική του νεότητα και
  γεωμετρική επεκτασιμότητα, ο νεοφλοιός δεν έχει υποστεί τόσες εξελικτικές βελτιστοποιήσεις, όσο τα υπόλοιπα τμήματα του εγκεφάλου,
  και διατηρεί σε μεγάλο βαθμό κοινή μορφολογία σε όλη του την έκταση. Αυτή η παρατήρηση ενδεχομένως να καθιστά το πρόβλημα
  της διάκρισης των θεμελιωδών αρχών λειτουργίας του από τις εξελικτικές βελτιστοποιήσεις πολύ ευκολότερο, σε σχέση με άλλα τμήματα.

  Ο δευτερεύων στόχος της HTM είναι αυτός με τον οποίο ασχολείται αυτή η εργασία.
  Συγκεκριμένα, η HTM μπορεί να χρησιμοποιηθεί για πρόβλεψη αιτιωδών ακολουθιών (πχ χρονοσειρών) και για αναγνώριση ανωμαλιών,
  ή γενικότερα για την αντιμετώπιση προβλημάτων πρόβλεψης ή κατηγοριοποίησης σε μη στάσιμες ροές δεδομένων.
  Έχει χρησιμοποιηθεί επιτυχώς ως τώρα (ως τεχνολογία βάσης κερδοσκοπικών επιχειρήσεων) για πρόβλεψη χρηματιστηριακών δεικτών
  και για έγκαιρη ανίχνευση ανωμαλιών σε κέντρα δεδομένων.

\section{Επιλογή της γλώσσας Julia}

  Για την παρουσιαζόμενη υλοποίηση της HTM επιλέχθηκε η σχετικά καινούρια γλώσσα επιστημονικού προγραμματισμού Julia \parencite{bezansonJuliaFreshApproach2017}.

  Προτού γίνει αυτή η επιλογή, δοκιμάστηκε η υλοποίηση της HTM σε Matlab.
  Όμως η Matlab αποτελεί ένα κλειστό οικοσύστημα, με λίγες προοπτικές για επαναχρησιμοποίηση και ευρύ απόηχο μιας τέτοιας δουλειάς,
  που εξαρχής στοχεύει στην υποβοήθηση περαιτέρω έρευνας.
  Η Matlab έχει μακρά ιστορία στο χώρο του επιστημονικού λογισμικού και γράφτηκε το 1984 στοχεύοντας ειδικά σε υπολογιστικούς επιστήμονες και όχι σε μηχανικούς λογισμικού.
  Πολλές σχεδιαστικές αποφάσεις ανακλούν αυτήν την εστίαση και δημιουργούν ένα περιβάλλον ανάπτυξης λογισμικού πιο δύσχρηστο σε σχέση με εναλλακτικές όπως η Python.
  Η γλώσσα διευκολύνει μεν τη χρήση γραμμικής άλγεβρας, αλλά δεν επιτρέπει συναρτησιακό προγραμματισμό, "lazy operations", ορισμό νέων τύπων δεδομένων
  και πολλά ακόμα στοιχεία απαραίτητα για την επιτυχία του κεντρικού στόχου αυτής της εργασίας: την εκφραστική απλότητα.
  Έτσι, η πρώτη υλοποίηση σε Matlab βοήθησε στην κατανόηση των αλγορίθμων, αλλά απέτυχε στον κεντρικό της στόχο.

  Η Julia είναι ανοιχτό λογισμικό που ξεκίνησε από το JuliaLab του MIT το 2012 και αναπτύσσεται δημοσίως στο Github.
  Μόλις τον Αύγουστο του 2018 έφθασε στην πρώτη επίσημη έκδοσή της.
  Οι δημιουργοί της δηλώνουν ως κίνητρο για τη δημιουργία της την αντιμετώπιση του "προβλήματος των 2 γλωσσών" στην επιστημονική υπολογιστική:
  μία γλώσσα υψηλού επιπέδου, εύχρηστη αλλά όχι τόσο αποδοτική όπως η Python, χρησιμοποιείται αρχικά για την κατασκευή μιας πρωτότυπης λύσης·
  έπειτα το λογισμικό ξαναγράφεται σε μια πιο δύσχρηστη, αλλά αποδοτική γλώσσα,
  ενδεχομένως κατάλληλη για HPC ή για να αξιοποιήσει παραδοσιακές υπολογιστικές συστοιχίες, όπως η C++.
  Αυτή η ροή εργασίας είναι σύνθετη, αργή και αναποτελεσματική, απαιτώντας διπλή προσπάθεια και, συχνά, εξειδικευμένο προσωπικό.
  Η Julia επιδιώκει να αποτελέσει λύση σε αυτό το πρόβλημα, συνδυάζοντας την ευχρηστία της Python και την αποδοτικότητα της C++.
  Αν και νέα γλώσσα, έχει ήδη χτίσει ένα πλούσιο οικοσύστημα για επιστημονικό προγραμματισμό.
  Συμπεριλαμβάνει προφανώς τα βασικά όπως γραμμική άλγεβρα και αραιούς πίνακες.
  Σε μερικά πεδία όμως, όπως η αστρονομία, οι διαφορικές εξισώσεις και τα πολύπλοκα συστήματα, προσφέρει ήδη εξίσου πλήρεις ή πληρέστερες λύσεις
  από πιο παραδοσιακές γλώσσες, όπως η Python.

  Μια επιτυχής υλοποίηση μεγάλου επιστημονικού λογισμικού που χαίρει ευρείας διαφήμισης για τη Julia είναι η Celeste \cite{regierLearningAstronomicalCatalog2016}.
  Γραμμένο σε Julia, χρησιμοποιεί παραλληλισμό κοινής και κατανεμημένης μνήμης και μπόρεσε να αξιοποιήσει
  8192 επεξεργαστικούς πυρήνες στον υπερυπολογιστή NERSC Cori.

\subsection{Σύντομη παρουσίαση της γλώσσας Julia}

  Στην πράξη, η Julia καθιστά εύχρηστο ένα μικτό προγραμματιστικό μοντέλο προστακτικού και συναρτησιακού προγραμματισμού.
  Η δηλωτική φύση του συναρτησιακού προγραμματισμού μένει πιστή στη μαθηματική διατύπωση της θεωρίας και αυξάνει σημαντικά την εκφραστικότητα του κώδικα,
  προτρέποντας την αποσύνθεση πολύπλοκων ορισμών σε απλούστερους.
  Απελευθερώνει δε το πρόγραμμα από την έμμεση σειριοποίηση που επιβάλλει ο προστακτικός προγραμματισμός και διευκολύνει την αναδιάταξη του κώδικα,
  όπως και την κλιμάκωση της εκτέλεσής του σε περισσότερους υπολογιστικούς πόρους ("παραλληλοποίηση").
  Ο προστακτικός προγραμματισμός οδηγεί σε ορισμένες περιπτώσεις σε πιο φυσική ή εύκολη διατύπωση της λύσης, ενώ σε άλλες λειτουργεί υπό μορφή βελτιστοποίησης.

  Ο σχεδιασμός της Julia προσπαθεί να διευκολύνει τον προγραμματιστή, αν επιθυμεί να υλοποιήσει τη λύση του με τον πιο πρόχειρο τρόπο,
  και να του επιτρέψει να τη βελτιώσει και να την κάνει αποδοτική με περισσότερη προσπάθεια και φροντίδα.
  Πιστή σε αυτήν την αρχή, υιοθετεί προαιρετικό σύστημα τύπων (δε θα ήταν άστοχη η σύγκριση με typed λ-calculus).

  Ένα πρόγραμμα Julia είναι κατά κανόνα ένα σύνολο ορισμών τύπων και τιμών. Στις τιμές μπορεί να αντιστοιχηθούν ονόματα, σύμβολα. Όταν δηλώνεται
  \begin{lstlisting}[language=julia]
  julia> x= 1
  1
  \end{lstlisting}
  η τιμή 1 αντιστοιχίζεται στο σύμβολο x.
  Έτσι ο τύπος του x είναι ο τύπος της τιμής του:
  \begin{lstlisting}[language=julia]
  julia> x|> typeof
  Int64

  julia> x= 1.0
  1.0

  julia> x|> typeof
  Float64
  \end{lstlisting}
  Σε όλην αυτήν την παρουσίαση δε χρησιμοποιήθηκε ο όρος "μεταβλητή". Το x είναι απλώς ένα σύμβολο, που εδώ αντιστοιχίστηκε σε μια σταθερή, αμετάβλητη τιμή,
  και μετά \textit{επαναντιστοιχίστηκε} σε μια διαφορετική σταθερή, αμετάβλητη τιμή.

  Το x θα μπορούσε να αντιστοιχιστεί και σε μια όντως μεταβλητή τιμή, δηλαδή μία τιμή που επιτρέπεται να τροποποηθεί:
  \begin{lstlisting}[language=julia]
  julia> x= rand(Int8,5)
  5-element Array{Int8,1}:
   31
    5
   86
   -3
   59

  julia> x[5]= 0
  0

  julia> x
  5-element Array{Int8,1}:
   31
    5
   86
   -3
    0
  \end{lstlisting}

  Εξίσου, η τιμή που συμβολίζει το x μπορεί να είναι ένας τύπος
  \begin{lstlisting}[language=julia]
  julia> x= Int64
  Int64

  julia> x|> typeof
  DataType
  \end{lstlisting}
  ή μια συνάρτηση
  \begin{lstlisting}[language=julia]
  julia> x= (i)-> i+1
  #3 (generic function with 1 method)

  julia> x|> typeof
  getfield(Main, Symbol("##3#4"))

  julia> x(5)
  6
  \end{lstlisting}

  Ο μεταγλωττιστής εσωτερικά αποδίδει τύπο σε κάθε τιμή που απαντά στο πρόγραμμα.
  Ο προγραμματιστής δε χρειάζεται να συσχετίσει ευθέως σύμβολα με τύπους.
  Μπορεί όμως, αν θέλει, να χρησιμοποιήσει τύπους για ένα βασικό σκοπό:
  "πολλαπλή αποστολή" (multiple dispatch) μεθόδων συνάρτησης βάσει τύπων.
  Οι όροι "μέθοδος" και "συνάρτηση" σημαίνουν διαφορετικά πράγματα στη Julia. Παραπάνω ορίσαμε μία ανώνυμη συνάρτηση που υλοποιείται από μία μέθοδο.
  Θα μπορούσαμε όμως να ορίσουμε και συνάρτηση που να μην υλοποιείται από καμία μέθοδο:
  \begin{lstlisting}[language=julia]
  julia> function myfun end
  myfun (generic function with 0 methods)

  julia> myfun()
  ERROR: MethodError: no method matching myfun()
  \end{lstlisting}

  Ας προσθέσουμε 2 μεθόδους στη συνάρτηση, για να φανεί η "πολλαπλή αποστολή" κι η χρήση τύπων:
  \begin{lstlisting}[language=julia]
  julia> myfun(i::Int)= print("I'm an Int")
  myfun (generic function with 1 method)

  julia> myfun(3)
  I'm an Int

  julia> myfun(i::Float64)= print("I'm a Double!")
  myfun (generic function with 2 methods)

  julia> myfun(3.0)
  I'm a Double!

  julia> myfun(3)
  I'm an Int

  julia> myfun()
  ERROR: MethodError: no method matching myfun()
  \end{lstlisting}

  Μέθοδος λοιπόν είναι ο συνδυασμός μίας συνάρτησης και μιας πλειάδας (tuple) ορισμάτων.
  Αυτή η σχεδίαση επιτρέπει έναν πολυμορφισμό συγκρίσιμο με της C++.

  Ο αναγνώστης θα κρίνει ο ίδιος την αποτελεσματικότητα των παραπάνω στον κορμό της εργασίας, όπου θα έρθει σε επαφή με αυτό το ιδίωμα γραφής.
