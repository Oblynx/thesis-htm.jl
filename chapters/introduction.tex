\section{Θεμελιώνοντας την έννοια της νοημοσύνης}

Κάθε φυσικό σύστημα καθορίζεται από τους περιορισμούς στα σύνορά του.
Ομοίως ο άνθρωπος καθορίζεται από την αλληλεπίδραση με το περιβάλλον του.
Στην προσπάθεια να κατανοήσουμε τον άνθρωπο, το πώς λειτουργεί, το γιατί δρα με συγκεκριμένο τρόπο,
σταθμό αποτελεί η κατανόηση της συμπεριφοράς που καλούμε \textit{νοημοσύνη}.

Η συμπεριφορά είναι παρατηρήσιμη και μας δίνει μια οπτική στην εσωτερική κατάσταση, στις κρυφές μεταβλητές ενός συστήματος,
οπότε ίσως αποτελεί το σημείο όπου μπορούμε να πιάσουμε το νήμα της αναζήτησης.
Είναι όμως ικανοποιητικό να χαρακτηρίσουμε τη νοημοσύνη συμπεριφορά;

Ο όρος νοημοσύνη χαίρει ευρείας ερμηνείας, ευρισκόμενος στο σταυροδρόμι πολλών επιστημονικών πεδίων, από την ψυχολογία μέχρι επιστήμη υπολογιστών
\parencite{leggCollectionDefinitionsIntelligence2007}.
Όλες μάλιστα οι πιο συγκεκριμένες ερμηνείες προσπίπτουν στο ιδιαίτερα ασαφές νόημα του όρου στην καθομιλουμένη.
Ίσως μπορούμε να ξεμπλέξουμε για πρακτικούς σκοπούς αυτό το κουβάρι, παρατηρώντας ότι νοημοσύνη σίγουρα επιδεικνύουν ζωντανοί οργανισμοί ως εξελικτικό χαρακτηριστικό,
ως εργαλείο στη διαρκή προσαρμογή τους στο επίσης δυναμικό τους περιβάλλον.

\subsection*{Νοημοσύνη ως προσαρμογή}

  Η νοημοσύνη λοιπόν, ως προσαρμογή, προκύπτει από τη σχέση του οργανισμού με το περιβάλλον του.
  Αν ο οργανισμός προσαρμόζεται ταιριάζοντας υλικά του κατασκευάσματα στο περιβάλλον του, η νοημοσύνη επεκτείνει αυτήν τη δημιουργικότητα, επιτρέποντάς του να πειραματιστεί εικονικά.
  \parencite[σελ 3]{piagetOriginsIntelligenceChildren1952}.
  Ας ακολουθήσουμε όμως τη σκέψη του ψυχολόγου Jean Piaget στο ζήτημα.

  Προσαρμογή είναι μια δυναμική διαδικασία αλληλεπίδρασης του οργανισμού με το περιβάλλον που περιλαμβάνει 2 στάδια: \textit{αφομοίωση} και \textit{συμβιβασμό}.
  Έστω ότι ο οργανισμός μπορεί να περιγραφεί με μια σειρά εσωτερικών μεταβλητών $\{a,b,c\}$ και εξωτερικών στοιχείων του περιβάλλοντος $\{x,y,z\}$,
  που συνδέονται μεταξύ τους με κάποιες διαδικασίες, ορίζοντας ένα μοντέλο (schema):
  \begin{align*}
    a + x &\rightarrow b\\
    b + y &\rightarrow c\\
    c + z &\rightarrow a
  \end{align*}

  Η \textit{αφομοίωση} συνίσταται στην ικανότητα του οργανισμού να ενσωματώνει τα στοιχεία του περιβάλλοντος στις εσωτερικές του καταστάσεις και να συνεχίζει αυτές τις διαδικασίες.
  Μια αλλαγή όμως στο περιβάλλον, έστω $x\rightarrow x'$, αποτελεί πρόκληση.
  Είτε ο οργανισμός δεν προσαρμόζεται, που σημαίνει ότι ο κύκλος σπάει και ο οργανισμός παύει να επιτελεί κάποια λειτουργία του,
  είτε προσαρμόζεται, τροποποιώντας υποχρεωτικά το μοντέλο του για να \textit{συμβιβαστεί} με τη νέα εξωγενή πραγματικότητα ($b\rightarrow b'$):
  \begin{align*}
    a  + x' &\rightarrow b'\\
    b' + y  &\rightarrow c\\
    c  + z  &\rightarrow a
  \end{align*}

  Σύμφωνα με αυτήν την περιγραφή, \textit{προσαρμογή} είναι η ισορροπία της \textit{αφομοίωσης} ($a+x\rightarrow b$...) με το \textit{συμβιβασμό} ($x\rightarrow x' \Rightarrow b\rightarrow b'$).

  Η προηγούμενη περιγραφή ισχύει εξίσου για τη \textit{νοημοσύνη}. Νοημοσύνη είναι αφομοίωση, στο βαθμό που συμπεριλαμβάνει όλα τα εμπειρικά δεδομένα στη δομή της.
  Είναι όμως και συμβιβασμός, καθώς κατά τη διαρκή αφομοίωση αποκρίνεται στην πρόκληση των περιβαλλοντικών αλλαγών με τροποποίηση του μοντέλου που περιγράφει τον κόσμο.

  Η διανοητική προσαρμογή λοιπόν, όπως κάθε προσαρμογή, συνίσταται από ένα μηχανισμό αφομοίωσης και συμπληρωματικού συμβιβασμού που διατηρούνται σε διαρκή ισορροπία.
  Ένα μυαλό προσαρμοσμένο στην πραγματικότητα είναι αυτό που δε δέχεται πια προκλήσεις στο νοητικό του μοντέλο για τον κόσμο,
  που δε χρειάζεται να τροποποιήσει περαιτέρω το μοντέλο αυτό για να εξηγήσει την εξελισσόμενη πραγματικότητα \parencite[σελ 5-7]{piagetOriginsIntelligenceChildren1952}.

\subsection*{Ένας πρακτικός ορισμός}

  Από το συμπεριφορικό ορισμό του Turing στο γνωστό "Turing test" μέχρι το μηχανιστικό ορισμό του Piaget,
  και με πολλές στάσεις ενδιάμεσα στο \cite[Lenat][]{lenatThresholdsKnowledge1991} και στο \cite[Minsky][]{minskySocietyMind1988},
  η πρακτικότητα της έννοιας τίθεται υπό αμφισβήτηση.

  Ένας άξονας του ορισμού είναι η σχέση της νοημοσύνης με τη λογική. Στο \cite{wangCognitiveLogicMathematical}
  ο Wang χωρίζει τα συλλογιστικά συστήματα σε 3 κατηγορίες:
  \begin{itemize}
    \item Αμιγώς αξιωματικά. Όλες οι λογικές προτάσεις προκύπτουν από τα αξιώματα, χαρακτηριστικό παράδειγμα η ευκλείδια γεωμετρία.
    \item Μερικώς αξιωματικά. Η γνώση δεν επαρκεί σε όλες τις περιπτώσεις και υπάρχει μηχανισμός προσαρμογής, όπως στα ασαφή συστήματα.
    \item Μη αξιωματικά. Χτίζονται με βάση την υπόθεση ότι η γνώση ή οι πόροι δεν επαρκούν για οποιοδήποτε συλλογισμό.
  \end{itemize}

  Προτείνει λοιπόν ότι η απαίτηση νοημοσύνης ισοδυναμεί με την απαίτηση μη αξιωματικού συλλογιστικού συστήματος,
  λόγω της ανεπάρκειας πληροφορίας και πόρων.

  Προσθέτοντας άλλο ένα έρεισμα στη συζήτηση, η νοημοσύνη συσχετίζεται άμεσα με τη δημιουργικότητα \parencite{benedekIntelligenceCreativityCognitive2014}.
  Δημιουργικότητα είναι η ικανότητα παραγωγής καινοτόμων και χρήσιμων ιδεών,
  επομένως συνδέεται με την τέλεση των προαναφερθέντων συμβιβασμών.
  \bigskip
  
  Σε αναζήτηση ενός χρήσιμου ορισμού στα πλαίσια αυτής της εργασίας, μπορούμε να στραφούμε στον ορισμό του \cite[Wang][]{wangWorkingDefinitionIntelligence1995}:

  \begin{displayquote}
    Νοημοσύνη είναι η ικανότητα ενός συστήματος επεξεργασίας πληροφορίας να προσαρμόζεται
    στο περιβάλλον του με ανεπαρκή γνώση και πόρους.
  \end{displayquote}

  Αναφερόμαστε σε σύστημα επεξεργασίας πληροφορίας, για να μπορούμε να μελετήσουμε την εσωτερική του κατάσταση και αλληλεπίδραση
  με το περιβάλλον αφηρημένα, σε αντιδιαστολή με ένα πρόβλημα π.χ. ρομποτικής.
  Το σύστημα έχει μια γλώσσα εισόδου και εξόδου, με την οποία εκφράζονται τα ερεθίσματα του περιβάλλοντος και οι δράσεις του συστήματος.
  Το σύστημα συνήθως έχει κάποιο σκοπό για τον οποίο παράγει δράσεις σύμφωνα με τη γνώση του,
  και για την επεξεργασία του δαπανά περιορισμένους πόρους.

  Η προσαρμογή μπορεί να ερμηνευθεί όπως προηγουμένως, ή πιο συνοπτικά ως ότι το σύστημα μαθαίνει από τις εμπειρίες του.

  Ο περιορισμός της ανεπαρκούς γνώσης και πόρων σημαίνει ότι το σύστημα υπόκειται σε αυτές τις συνθήκες:
  \begin{itemize}
    \item \textit{Περιορισμένο ως προς τους υπολογιστικούς του πόρους}
    \item Λειτουργεί σε \textit{πραγματικό χρόνο}
    \item Καλύπτει όλο το πεδίο των εκφράσιμων στη γλώσσα του εισόδων και εξόδων (\textit{δεν υπάρχουν άκυρες είσοδοι/έξοδοι})
  \end{itemize}

  Σύμφωνα με αυτόν τον ορισμό νοημοσύνη είναι μια ισχυρή μορφή προσαρμογής.

  Οι παραπάνω περιορισμοί μας οδηγούν προς την εξέταση \textit{συστημάτων ροής (streaming)}, με \textit{ανθεκτικότητα σε σφάλματα}
  και με \textit{απόδοση που να κλιμακώνεται με τους διαθέσιμους πόρους} --- υπό την προϋπόθεση της \textit{διαρκούς προσαρμογής σε νέες συνθήκες} του περιβάλλοντος.

  Η αυλαία σηκώνεται για να παρουσιαστεί το υπό μελέτη σύστημα: \textbf{Hierarchical Temporal Memory (HTM) της Numenta}.
