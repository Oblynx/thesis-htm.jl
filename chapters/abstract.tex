\thispagestyle{empty}
\begin{center}
   \Large
   \textbf{\titlestring}

   \vspace{0.4cm}
   \large
   %Thesis Subtitle

   \vspace{0.4cm}
   \textbf{\authorstring}

   \vspace{0.9cm}
   \textbf{\abstractname}
\end{center}

%\begin{abstract}
    Παρουσιάζεται μια αλγοριθμική θεωρία της νοημοσύνης εν τη γενέσει, η \textbf{Hierarchical Temporal Memory} (HTM).
    Βασικοί αλγόριθμοι της θεωρίας υλοποιούνται σε υψηλού επιπέδου γλώσσα με εκφραστική περιεκτικότητα, τη Julia,
    για την καταπολέμηση της υψηλής \textbf{γλωσσικής πολυπλοκότητας} (δυσκολίας στην περιγραφή).
    Η προγραμματιστική διατύπωση των αλγορίθμων παραμένει κοντά στην πηγαία μαθηματική διατύπωση,
    διευκολύνοντας τη μετέπειτα ανάλυση και τον πειραματισμό με νέες αλγοριθμικές ιδέες και προεκτάσεις.
    Η HTM είναι βιολογικά περιορισμένη θεωρία, που αποσκοπεί καταρχήν στην εξήγηση της λειτουργίας του νεοφλοιού (εγκεφαλική δομή)
    και μόνο κατ'επέκτασιν σε εφαρμογές τεχνητής νοημοσύνης.
    Οι υλοποιήσεις σε Julia περιγράφονται αναλυτικά.
    Επαληθεύονται με τις υλοποιήσεις αυτές βασικές ιδιότητες των αλγορίθμων, εν είδει ελέγχου ορθότητας.
    Εν τέλει, προτείνονται κατευθύνσεις έρευνας στην HTM που διευκολύνονται από την παρούσα εργασία.
    Κορωνίδα του έργου είναι η \textit{δημοσίευση πακέτου ανοιχτού λογισμικού} που υλοποιεί τους βασικούς αλγορίθμους HTM σε Julia.
%\end{abstract}

\begin{center}
    \large
    \vspace{2.9cm}
    \textbf{Abstract}
\end{center}

\EN{
    An algorithmic theory of intelligence under development is presented, the \textbf{Hierarchical Temporal Memory} (HTM).
    Fundamental algorithms of the theory are implemented in an expressive high level language, Julia,
    to counteract their high \textbf{linguistic complexity} (difficulty in describing).
    The programmatic definition of the algorithms stays faithful to the source mathematical definition,
    benefitting further analysis and experimentation with novel algorithmic ideas and extensions.
    HTM is a biologically constrained theory, aiming primarily to model the function of the neocortex (brain structure)
    and as a secondary goal to artificial intelligence applications.
    The Julia implementations are described in detail.
    Fundamental properties of the algorithms are corroborated with this implementation as a basic correctness test.
    Finally, directions for further study of HTM theory facilitated by the present work are highlighted.
    The crowning achievement is the \textit{publication of an implementation of the fundamental HTM algorithms in Julia as open source software}.
}
