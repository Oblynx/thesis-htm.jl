\thispagestyle{plain}
\begin{center}
    \Large
    \textbf{\titlestring}

    \vspace{0.4cm}
    \large
    %Thesis Subtitle

    \vspace{0.4cm}
    \textbf{Κωνσταντίνος Σαμαράς-Τσακίρης}

    \vspace{0.9cm}
    \textbf{Abstract}
\end{center}

Παρουσιάζεται μια αλγοριθμική θεωρία της νοημοσύνης εν τη γενέσει, η \textbf{Hierarchical Temporal Memory} (HTM).
Βασικοί αλγόριθμοι της θεωρίας υλοποιούνται σε υψηλού επιπέδου γλώσσα με εκφραστική περιεκτικότητα, τη Julia,
για την καταπολέμηση της υψηλής \textbf{γλωσσικής πολυπλοκότητας} (δυσκολίας στην περιγραφή).
Η προγραμματιστική διατύπωση των αλγορίθμων παραμένει κοντά στην πηγαία μαθηματική διατύπωση,
διευκολύνοντας τη μετέπειτα ανάλυση και τον πειραματισμό με νέες αλγοριθμικές ιδέες και προεκτάσεις.
Η HTM είναι βιολογικά περιορισμένη θεωρία, που αποσκοπεί καταρχήν στην εξήγηση της λειτουργίας του νεοφλοιού (εγκεφαλική δομή)
και μόνο κατ'επέκτασιν σε εφαρμογές τεχνητής νοημοσύνης.
Οι υλοποιήσεις σε Julia περιγράφονται αναλυτικά.
Επαληθεύονται με τις υλοποιήσεις αυτές βασικές ιδιότητες των αλγορίθμων, εν είδει ελέγχου ορθότητας.
Εν τέλει, προτείνονται κατευθύνσεις έρευνας στην HTM που διευκολύνονται από την παρούσα εργασία.
Κορωνίδα του έργου είναι η \textit{δημοσίευση πακέτου ανοιχτού λογισμικού} που υλοποιεί τους βασικούς αλγορίθμους HTM σε Julia.
